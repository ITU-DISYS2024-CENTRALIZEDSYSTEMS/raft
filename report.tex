% report.tex

\documentclass[a4paper,11pt]{article}
\usepackage{hyperref}
\usepackage{enumitem}
\usepackage{varwidth}
\usepackage{tasks}
% Import packages
\usepackage[a4paper]{geometry}
\usepackage[utf8]{inputenc}
\usepackage{amsmath}
\usepackage{amssymb}
\usepackage{enumerate}
\usepackage{geometry}
 \geometry{
 a4paper,
 total={170mm,257mm},
 left=20mm,
 top=20mm,
 }

\usepackage{graphicx}

\usepackage{listings}


% Change enumerate environments you use letters
\renewcommand{\theenumi}{\alph{enumi}}

% Set title, author name and date
\title{Raft}
\author{Johannes Jørgensen (jgjo),\\ Kevin Skovgaard Gravesen (kegr),\\ Joakim Andreasen (joaan)} 
\date{\today}

\begin{document} 

\maketitle

\subsection*{Introduction}

\subsection*{Go features}
The raft implementation is written using go routines and some channels.
Some examples of the go routines being used could be the appending and commiting of information to the log, election handling and heartbeats.
The reason for the use of channels is to move the decisions and actions take by the goroutines to the main routine.
By combining these go features as much of the Raft implementation is multithreaded as possible.

Go's defer keyword is also used but, besides the routine and channel usage, the rest of the go features is built in go quality of life features.
Which is Go's spin on features from other languages, such as slices.

\subsection*{Node communication}
The communication between nodes is done through the use of proto-buff and GRPC, where the nodes can send and recieve votes. It is also used to send heartbeats. 
The program also communicates internally between the goroutines, and is done through channels. This is to ensure flexibility and responsiveness of the the individual nodes. 

\subsection*{Implementation quality}

\subsubsection*{Election safety}
A constant running "electionTicker" go routine is being send to sleep with a random amount of time.
This is to ensure that no two nodes are asking for an election at the same time.
The "mutex" structure is also used for atomic locks which ensures that each node's internal decisions can be trusted.
And that a node for example is not both accepting another node to be elected and voting on it self at the same time.

\subsubsection*{Leader Append-Only}
Multiple checks are in place to ensure that only the leader could append to the log.
That can, among other places, be seen in the "sendAppendEntries" function that send new entries to the other nodes.
This function have a check in it that prevents to node from sending anything if it is not the leader.
With this even a wrongly use of the method won't cause an issue in this case, even tho it should be rectified.

The majority requirement of adding to the log is also implemented, 
which Raft requires to ensure that the leader is not deciding on decisions without the other nodes following along.

\subsubsection*{Log Matching}

\subsubsection*{Leader completeness}
To ensure a leader is either active or being actively selected, without ending in deadlock, the implementation uses a randomizer that delays the vote request for a random amount of time (default 300ms and 600ms).
The randomizer is implemented to randomize who will send the request for votes first and therefore the chances of the same leader being chosen every program restart. The program also uses majority vote system,
to mitigate the chance of a deadlock when there are two nodes getting votes. If there is two nodes that get equal amount of votes, the nodes resets the election and sends new requests for votes.
Then due to the randomizer, the chances of a non-majority vote happens diminishes for every election reset that occurs.

\subsubsection*{State Machine Safety}

\subsection*{Source}
\href{https://github.com/jmsadair/raft}{https://github.com/jmsadair/raft}

\end{document}
