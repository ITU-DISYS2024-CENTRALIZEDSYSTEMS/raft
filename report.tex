% report.tex

\documentclass[a4paper,11pt]{article}
\usepackage{hyperref}
\usepackage{enumitem}
\usepackage{varwidth}
\usepackage{tasks}
% Import packages
\usepackage[a4paper]{geometry}
\usepackage[utf8]{inputenc}
\usepackage{amsmath}
\usepackage{amssymb}
\usepackage{enumerate}
\usepackage{geometry}
 \geometry{
 a4paper,
 total={170mm,257mm},
 left=20mm,
 top=20mm,
 }

\usepackage{graphicx}

\usepackage{listings}


% Change enumerate environments you use letters
\renewcommand{\theenumi}{\alph{enumi}}

% Set title, author name and date
\title{Raft}
\author{Johannes Jørgensen (jgjo),\\ Kevin Skovgaard Gravesen (kegr),\\ Joakim Andreasen (joaan)} 
\date{\today}

\begin{document} 

\maketitle

\subsection*{Introduction}

\subsection*{Go features}
The raft implementation is written using go routines and some channels.
Some examples of the go routines being used could be the appending and commiting of information to the log, election handling and heartbeats.
The reason for the use of channels is to move the decisions and actions take by the goroutines to the main routine.
By combining these go features as much of the Raft implementation is multithreaded as possible.

\subsection*{Node communication}

\subsection*{Implementation quality}

\subsubsection*{Election safety}

\subsubsection*{Leader Append-Only}

\subsubsection*{Log Matching}

\subsubsection*{Leader completeness}

\subsubsection*{State Machine Safety}

\subsection*{Source}
\href{https://github.com/jmsadair/raft}{https://github.com/jmsadair/raft}

\end{document}