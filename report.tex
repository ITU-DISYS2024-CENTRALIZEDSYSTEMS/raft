% report.tex

\documentclass[a4paper,11pt]{article}
\usepackage{hyperref}
\usepackage{enumitem}
\usepackage{varwidth}
\usepackage{tasks}
% Import packages
\usepackage[a4paper]{geometry}
\usepackage[utf8]{inputenc}
\usepackage{amsmath}
\usepackage{amssymb}
\usepackage{enumerate}
\usepackage{geometry}
 \geometry{
 a4paper,
 total={170mm,257mm},
 left=20mm,
 top=20mm,
 }

\usepackage{graphicx}

\usepackage{listings}


% Change enumerate environments you use letters
\renewcommand{\theenumi}{\alph{enumi}}

% Set title, author name and date
\title{Raft}
\author{Johannes Jørgensen (jgjo),\\ Kevin Skovgaard Gravesen (kegr),\\ Joakim Andreasen (joaan)} 
\date{\today}

\begin{document} 

\maketitle

\section*{Introduction}
The following report describes the Implementation of the Raft consensus algorithm in Go made by Github User \href{https://github.com/jmsadair/}{@jmsadair}.


Source: \href{https://github.com/jmsadair/raft}{Github Repository - Raft @jmsadair}.
\\

\section*{Go features}

\section*{Node communication}
@jmsadair uses gRPC as the method of communication between the nodes in the cluster. The \href{https://github.com/jmsadair/raft/blob/dev/internal/protobuf/raft.proto}{gPRCs} include the the ability of a node to 1. request missing log enteries from the leader 2. Request votes from other nodes in the cluster in a leader election. 


The \href{https://github.com/jmsadair/raft/blob/dev/transport.go}{transport.go} file handles the communication used by a node in the cluster, both to send and receive RPCs.

\section*{Implementation quality}
% This should be split into multiple subsubsections

\subsection*{Source}

\end{document}